\documentclass[color,english,personal]{ubletter}
\usepackage{siunitx}

\begin{document}
\begin{letter}{}
\subject{Submission of revised manuscript PONE-D-23-10896}

\opening{Dear Michael Schubert, Jay Richard Stauffer, Jr.\ and anonymous reviewer \#2}

We thank you for the opportunity to resubmit a revised version of our manuscript on the \emph{Microtomographic investigation of a large corpus of cichlids}.
We have implemented your suggestions and comments and considerably improved the manuscript.

Below, you will find our point-by-point response to the raised issues.

\begin{quote}
Reviewer \#1: Excellent paper. It was not clear if you could discern
daily rings on the otoliths. You might want to discuss if you can use
rings on otoliths to age the fish without removing them. Also, do you
have to stain the fish to use these rings?
\end{quote}

The data shown in this manuscript is from unstained specimen. They have
been stored in ethanol and imaged in wet foam to avoid any drying during
imaging, but \emph{no} staining was applied. Since we here only focus on
the morphology of the skull and teeth such a staining was not necessary
to extract the needed data from the tomographic images. Growth rings
would most probably be visible in the analyzed specimens, but only if
some staining is applied, e.g.~by storing the fish in Lugol solution.
Since this was \emph{not} the focus of the present study we did not do
this here.

To use the data on both growth rings and volume of the otoliths one
would have to calibrate this first to be able to extract age estimates.
I.e.\ one would have to make a simple growth experiment where a number of
fish with known age are euthanized, tomographically imaged and then the
volume of their extracted otoliths is correlated with their age. Such a
calibration would match size and volume of both the (lab-raised) fish
and the otoliths with the age of the fish. This data could then be used
to estimate the age of fishes collected in the wild.

\begin{quote}
Reviewer \#2: I have carefully evaluated the manuscript titled
``Microtomographic investigation of a large corpus of cichlids''. The
authors present a substantial collection of tomographic images from
cichlids collected in Lake Victoria, Africa. They showcase significant
cranial, maxillary, and otolith images from 362 specimens and provide
open-source Python notebooks to facilitate the manipulation and
extraction of specific parts. I believe that this work represents a
significant contribution to the understanding of cichlids and the
application of Micro-CT imaging in fish studies. However, I have some
suggestions to improve the manuscript:
\end{quote}

Thank you very much for the kind words. As we (now) state in the
\href{https://habi.github.io/EAWAG-manuscript/\#micro-computed-tomographic-imaging}{imaging
section} we acquired 372 single tomographic scans of 133 different
specimen (updated since submission of the manuscript, as a rest of fish
relevant for the study was added). This means that each specimen was
scanned several times, usually one lower-resolution scan of the full
head and two single higher-resolution scans for the oral and pharyngeal
jaws.

\begin{quote}
The introduction section needs improvement. It is too simplistic and
lacks informative details about the state of the art and the
justification for the work.
\end{quote}

We've deliberately kept the introduction rather terse. The justification
of the work lies in the non-destructive imaging of unstained samples. We
have expanded the introduction a bit and added references to manuscripts
detailing microtomographic imaging in the life sciences. As this
manuscript is not a review, we kindly refuse to add more information on
the background of microtomographic imaging.

\begin{quote}
Merging the Materials and Methods section with the Results is not a
reasonable way to present the work. It would be better to separate them
for clarity.
\end{quote}

The manuscript is focused on the method on how to acquire and assess
tomographic datasets of a large corpus of cichlids. Especially for the
image processing part, the method and results are highly intertwined as
the Python code directly delivers the results. We believe that the
approach we chose here here is the most appropriate way to present the
whole data data acquisition and analysis pipeline and kindly ask the
reviewer to accept our choice.

\begin{quote}
The section on Micro-computed tomographic imaging should be the core of
the paper and explained in more detail. The authors have extensive
experience scanning 362 fish specimens of different species and sizes.
Therefore, a detailed imaging protocol that can assist readers in
obtaining new images should be provided. For example, what criteria did
the authors use to select the best set of parameters? It is clear and
expected that larger fish would have lower resolution (larger voxel
size), but how does the relationship between size and parameters work?
How was the filter thickness chosen?
\end{quote}

As stated above and in the text, we acquired 372 single tomographic
scans of 133 different specimen.

\begin{quote}
The first sentence of the last paragraph of the subsection
Micro-computed tomographic imaging is unclear: ``While performing the
work, a subset of the data was always present on the production system,
for working with it (see Preparation for analysis below)''. Please
provide more precise information.
\end{quote}

We are sorry for the imprecise description of the data archival and
copying process. The last paragraph of
\href{https://habi.github.io/EAWAG-manuscript/\#micro-computed-tomographic-imaging}{this
section} was expanded and improved.

\begin{quote}
The authors should revise the entire ``Data analysis'' subsection to
present a more didactic version that helps any reader use the notebooks.
Currently, it seems that only Python users with some expertise can
understand what needs to be done. Perhaps the authors could provide a
guide or tutorial in the GitHub repository.
\end{quote}

The notebooks are written in Python, and are presented as Jupyter
notebooks. This makes them accessible `piece wise', e.g.~each part of
the code is didactically separate and somehow compartmentalized. The
code is written in Python, so some experience with it is beneficial to
thoroughly understand it. The code and notebooks are also prepared in a
way that \emph{no} installation of \emph{any} software except an up-to
date web browser is necessary to test it. This testing can be done by
clicking one button in the code repository (linking to
https://mybinder.org/v2/gh/habi/eawag/HEAD) to launch a Jupyter
environment in the cloud, which is also already explained in the last
paragraph before the
\href{https://habi.github.io/EAWAG-manuscript/\#discussion}{Discussion
section}. We understand the concerns of the reviewer though, and added
explanation text and comments to the notebooks guiding (first time)
users of the notebooks. All these updates happen in
\href{https://github.com/habi/EAWAG}{the code repository} and have
\emph{not} been specifically mentioned in the manuscript text. Two of
the notebooks
(\href{https://nbviewer.org/github/habi/EAWAG/blob/main/DataWrangling.ipynb}{DataWrangling}
and
\href{https://nbviewer.org/github/habi/EAWAG/blob/main/DataSize.ipynb}{DataSize})
were used to help us generate `checking' and `collaboration' files
(DataWrangling) or used to help us generate text for the manuscript,
i.e.~to extract exact details on sample numbers, total amount of
projections and reconstructions and their exact size on disk (DataSize).
As these notebooks are not explicitly mentioned in the manuscript they
are not extensively commented (only commented). The two notebooks
mentioned in the text
(\href{https://nbviewer.org/github/habi/EAWAG/blob/main/DisplayFishes.ipynb}{DisplayFishes})
and
(\href{https://nbviewer.org/github/habi/EAWAG/blob/main/ExtractOtoliths.ipynb}{ExtractOtoliths})
have been updated with more comments and explanatory text between the
logical steps. These additions are
\href{https://github.com/habi/EAWAG/compare/v1.1...HEAD}{visualized in
the respective GitHub repository}, but not specifically mentioned in the
text. We slightly expanded the description of the notebooks at the start
of the
\href{https://habi.github.io/EAWAG-manuscript/\#data-analysis}{Data
analysis section} though.

\begin{quote}
How was the sanity check performed (as mentioned in the second paragraph
of the subsection ``Preparation for Analysis'')?
\end{quote}

The mentioned `sanity check' was performed manually, e.g.~by looking at
all the values (mostly the reconstruction parameters), to exclude any
operator error. If errors were spotted at this stage, the
reconstructions were deleted and re-done with the correct parameters.
Sometimes errors or problems were only spotted at a later stage, when
either the pharyngeal or oral jaws were extracted, or when we noticed
that some part of the the whole skull were out of the field of view of
the tomographic scan prior to performing the PCA analysis mentioned
later in the manuscript. This means that specifying this as `sanity
check' is more of a glorified wording of telling the readers that we
manually looked at the values of each performed tomographic scan to
exclude and catch operator errors. We reworded the sentence to hopefully
make this more clear.

\begin{quote}
In the same paragraph, the authors mentioned that each tomographic
dataset contains around 2700 slices, exceeding the available RAM size on
an average high-end workstation. Could you clarify if this limitation
affected the analysis and its implications?
\end{quote}

We wrote that \emph{the total size of the acquired data exceeds the RAM}
of any workstation (e.g.~around 1.5 TB). \emph{One} or more single
datasets would fit into RAM very well, It is not possible to load
\emph{all} data concurrently for analysis. The use of \emph{Dask} and
more specifically \href{@https://image.dask.org/}{\emph{dask-image}} for
loading only the currently used data for each specimen from the HD (or
remote storage) into RAM on demand made working with the \emph{all} the
data feasible. We have updated the sentence to explain this.

\begin{quote}
The Wikipedia citation in the second paragraph of the subsection
``Extraction of oral and pharyngeal jaws\ldots{}'' is not clear. Please
provide more specific information or replace the citation with a more
appropriate reference.
\end{quote}

The Wikipedia citation to a \href{https://w.wiki/5mBK}{date-specific
version of the \texttt{Nrrd} article on Wikipedia} was not correctly
expanded by our manuscript preparation system. We replaced the Wikipedia
link with a more appropriate link to
\href{https://teem.sourceforge.net/nrrd/format.html}{the \texttt{Nrrd}
format definition} and thank the reviewer for noticing this issue.

\begin{quote}
Regarding the sentence ``In total, we compiled an overview of 125
specimens with full head morphology, oral jaw, and lower pharyngeal jaw
profiles,'' why was this done only for a subset of the sample? Please
explain the rationale.
\end{quote}

As stated above and in the text, we actually acquired 372 single
tomographic scans of 133 different specimen. It is correct though that
``125'' here is wrong, since we've acquired more scans after submission
of the manuscript. This has been corrected, and we thank the reviewer
for noticing this.

\begin{quote}
The results of the subsection ``Principal components analysis of skull
landmarks'' should be presented in a more accessible manner. Consider
providing a tutorial or guide for readers to better understand and apply
the analysis. As it currently stands, this subsection may be considered
irrelevant for the paper.
\end{quote}

This subsection is currently being `expanded' into a self-contained
publication. We deliberately only very briefly mentioned the method
here, as the upcoming publication is to be submitted soon.

\begin{quote}
The procedure for detecting and extracting otoliths in the subsection
``Automatic extraction of otoliths'' is not clear. Please provide
clearer instructions or guidelines for readers using the notebook.
\end{quote}

As mentioned above, the Jupyter notebook used to detect the otolith
position and to extract the otoliths from the scans covering the whole
skull are extensively commented. We have added more explanatory comments
and text to the notebook on GitHub, increasing their `didactiveness'.

\begin{quote}
Use the full words, rather than abbreviations, in figure legends for
clarity.
\end{quote}

Abbreviations in the legends have been written-out (except SI units).
Additionally we added a bit of background information on the specimen to
the legend of Fig. 1.

\begin{quote}
In the discussion section, the authors mentioned acquiring
high-resolution tomographic datasets of a large collection of cichlids.
However, the statement that ``the acquired datasets were imaged over a
wide-spanning range of voxel size (3.5--\SI{50}{\micro\meter})'' is incorrect. The voxel
size is specimen-specific, depending on the size of the fish and the
chosen parameters. Please clarify this point to accurately represent the
imaging process. The finest resolution was obtained only for small
fishes, so specific structures of a fish will have the same image
resolution throughout the entire specimen.
\end{quote}

The statement over the `wide-spanning range of voxel sizes' is
\emph{not} incorrect, at most---if applied to \emph{one}
specimen---imprecise. The range of voxel sizes is true for the whole
study, while---as also correctly noted by the reviewer---the voxel size
is specimen- or scan-specific. We have added a bit of explanatory text
to both the materials and discussion section.

\begin{quote}
It would be beneficial to include references in the discussion section
to support the arguments and provide a broader context for the findings.
\end{quote}

The main scientific `use case' of the method focuses on the automated,
non-destructive extraction of the otolith volume from the tomographic
data. To the best of our knowledge, our manuscript is the first to
describe such a method. We have found two references describing otoliths
or otolith structure, both on explanted otoliths. We have added those to
the discussion section.

\begin{quote}
Overall, the study seems to have spanned a considerable time frame.
However, the manuscript's content and structure need improvement to
reach a wider audience and effectively communicate the significance of
the work. The authors have provided a valuable set of open-source Python
notebooks, which is commendable and will be highly useful for the
scientific community. However, the workflow needs to be better explained
to enable readers to follow the procedures accurately.
\end{quote}

As mentioned above, we have improved both the text and the Jupyter
notebooks (linked in the text) with explanatory sentences and comments.

\begin{quote}
By addressing these suggestions, the manuscript will become more
comprehensive, accessible, and representative of the valuable
contribution made by the authors.
\end{quote}

We thank the reviewer for the kind words and the time spent on improving
our work with relevant questions and comments. The manuscript has
considerably improved by the updates we have performed.

We are looking forward to all your feedback if there is any remaining.

\addtolength{\medskipamount}{-\medskipamount} % skip some space with imported image
\closing{Yours sincerely\\\includegraphics[scale=0.618]{_Signature}}
\end{letter}
\end{document}
